% !TEX root = ./main.tex

\section{Project Introduction}\label{sec:project}

\repoimages{} is a \langjulia image-processing toolbox, including several packages such as: \imagecore, \imagetransformations, \imageaxes. It provides a collection of out-of-box functions\footnote{An overview of currently implemented image-processing functionalities is shown at \apicomparison.} to do image processing tasks just like \reposcikitimage and \matlabimageprocessing.

However, despite of the performance, this toolbox at present is still not friendly to both users and developers; Unlike other mature julia packages such as \repojump and \repogpuarrays, \images{} requires potential users to understand the very details of its mechanism and architecture, and this is even harder for them without comprehensive documentation on it. Under this circumstance, most image-processing researchers are still using \langpython and \langmatlab for their daily work.

Some apparent causes for its poor usability are:
\begin{itemize}
    \item there's few demos or recipes in \images{} for new users to start with;
    \item the APIs lack of consistence and don't match the julian style well;
    \item there's no style guide on naming and programming;
    \item there're too many temporary helper functions defined everywhere;
    \item \images{} is an ecosystem but it lacks of a comprehensive illustration of its packages;
    \item coverage of trait functions are not fully tested.
\end{itemize}
Fundamentally this is because that it is still in the progress of finding the most suitable programming style to process images using \langjulia.

Fortunately this problem is well-concerned in the community. Issues such as
\begin{itemize}
    \item \issue{\juliaimages}{\images}{766}
    \item \issue{\juliaimages}{\images}{767}
    \item \issue{\juliaimages}{\images}{772}
    \item \issue{zygmuntszpak}{\imagebinarization}{23}
\end{itemize}
dicuss the coding styles in the most generic way, and packages such as
\begin{itemize}
    \item \repohistogramthresholding
    \item \repoimagebinarization
\end{itemize}
are examples validating the effectiveness of style consensus reached in those issues. In \imagebinarization, one could binarize an image using any implemented methods\footnote{At the time of writing, there're 12 methods implemented.} with one unified API: \mint{julia}|binarize(::BinarizationAlgorithm, ::AbstractArray{T,2}) where {T}|

With these existing work, it's in the right time to revisit the whole \images ecosystem and head towards a more easy-to-use \images package. This project aims to solve this problem by:

\begin{itemize}
    \item providing more comprehensive and integrated documentation on both style guide and ecosystem illustration,
    \item pruning codebase of the ecosystem according to the provided documentation, and
    \item converting \images{} to an entrance to the ecosystem.
\end{itemize}

Writing demos of \images{} is not included in this project since it belongs to a totally different project. Basically, this is a project on documentation and code refactoring.
