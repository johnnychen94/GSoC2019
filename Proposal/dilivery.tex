% !TEX root = ./main.tex

\section{Delivery and milestone}\label{sec:delivery}

\todo{I can't estimate the timeline.}

As described in the end of \cref{sec:project}, the project will be delivered in two stages: drafting a style guide in the first stage as a preperation, and cleaning the codebase in the second stage.

\subsection{Documentation and Guide}\label{subsec:documentation}

The main purpose of this stage is to provide trackable documents for the next stage's pruning work. \emph{The documents will be in three forms: ecosystem documentation, developer manual, and RFC}. Ecosystem documentation illustrates the scope of image ecosystem and relationships between different relavent packages, it helps users and developers to understand what package belongs to \images{} and what package doesn't. Developer manual consists of style guide and best practice as well as other related community-operating rules. According to these two documents, a RFC with a detailed list of API changes and operation will be proposed, which will be stick onto in the next stage's pruning work.

To reach the milestone, following work will be done:

\begin{enumerate}
    \item dig into the source code of image-processing-related \langjulia packages as well as other mature \langjulia{} packages such as \repojump to filter out good and bad julia programming practices.
    \item compare the APIs of \images{} with that of \reposcikitimage{} and \matlabimageprocessing{} to find the balance between the native julian style and the familiar style for users from other languages.
    \item discuss with the cummunity and draft the ecosystem documentation and style guide, define the scope of the ecosystem.
    \item dig into the source code of \images{} ecosystem, and draft the RFC\footnote{Tensorflow community provides a \href{https://github.com/tensorflow/community/blob/master/rfcs/yyyymmdd-rfc-template.md}{RFC Template} to begin with, and \href{https://github.com/tensorflow/community/blob/master/rfcs/20180827-api-names.md}{20180827-api-names.md} is a good example.}
    \item discuss with the community and make the RFC approved.
\end{enumerate}

The approval of RFC indicates the end of this stage.

\subsection{Prune Codebase}\label{subsec:prune}

With the RFC approved, the prune stage is to clean the codebase following the RFC operation guide. A milestone will be set in \repoimages{} to track the progress.

60\% completeness rate shall be enough to indicate the end of this stage, since many repositories will be involved in this stage.

\todo{I don't know what should be take care of in this stage}
