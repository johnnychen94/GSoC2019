% !TEX root = ./main.tex

\documentclass[12pt, a4paper]{article}
\usepackage[utf8]{inputenc}
\usepackage[affil-it]{authblk}
\usepackage[colorlinks=true,urlcolor=blue,citecolor=SpringGreen4,bookmarks=true]{hyperref}
\usepackage{marvosym} % \Email
\usepackage{cleveref}
\usepackage{xspace}
\usepackage{minted}

% !TEX root = ./main.tex

\newcommand{\todo}[1]{\textcolor{red}{ TODO: (#1)}}
\newcommand{\ask}[1]{\textcolor{red}{ Question: #1}\par}

\newcommand{\mailto}[1]{\href{mailto:#1}{\textsuperscript{\Email}}}
\newcommand{\version}[1]{v$#1$\xspace}

\newcommand{\sname}[1]{\textsf{#1}} % special name

% programming
\newcommand{\langmatlab}{\sname{MATLAB}\xspace}
\newcommand{\langjulia}{\sname{Julia}\xspace}
\newcommand{\langpython}{\sname{Python}\xspace}

% github
\newcommand{\pullrequest}[3]{\href{https://github.com/#1/#2/pull/#3}{\sname{#1/#2\##3}}}
\newcommand{\issue}[3]{\href{https://github.com/#1/#2/pull/#3}{\sname{#1/#2\##3}}}

% organization names
\newcommand{\juliaimages}{JuliaImages}
\newcommand{\julialang}{JuliaLang}
\newcommand{\fluxml}{FluxML}
\newcommand{\juliamath}{JuliaMath}

% julia repository names
\newcommand{\reponame}[1]{#1}
\newcommand{\imagetransformations}{\reponame{ImageTransformations.jl}}
\newcommand{\imagecore}{\reponame{ImageCore.jl}}
\newcommand{\imageaxes}{\reponame{ImageAxes.jl}}
\newcommand{\imagemetadata}{\reponame{ImageMetadata.jl}}
\newcommand{\imagefiltering}{\reponame{ImageFiltering.jl}}
\newcommand{\imagefeature}{\reponame{ImageFeature.jl}}
\newcommand{\imagemorphology}{\reponame{ImageMorphology.jl}}
\newcommand{\imagetracking}{\reponame{ImageTracking.jl}}
\newcommand{\images}{\reponame{Images.jl}}
\newcommand{\flux}{\reponame{Flux.jl}}
\newcommand{\jump}{\reponame{JuMP.jl}}
\newcommand{\gpuarrays}{\reponame{GPUArrays.jl}}
\newcommand{\pycall}{\reponame{PyCall.jl}}
\newcommand{\imagebinarization}{\reponame{ImageBinarization.jl}}
\newcommand{\histogramthresholding}{\reponame{HistogramThresholding.jl}}
\newcommand{\imagenoise}{\reponame{ImageNoise.jl}}
\newcommand{\imagedistance}{\reponame{ImageDistances.jl}}
\newcommand{\imagesgithubio}{\reponame{juliaimages.github.io}}
\newcommand{\testimages}{\reponame{TestImages.jl}}
\newcommand{\fixedpointnumbers}{\reponame{FixedPointNumbers.jl}}
\newcommand{\imageinpainting}{\reponame{ImageInpainting.jl}}
\newcommand{\imageview}{\reponame{ImageView.jl}}


% repository urls
\newcommand{\gitrepo}[2]{\href{https://github.com/#1/#2}{\sname{#2}}\xspace}
\newcommand{\repoimagetransformations}{\gitrepo{\juliaimages}{\imagetransformations}}
\newcommand{\repoimagecore}{\gitrepo{\juliaimages}{\imagecore}}
\newcommand{\repoimageaxes}{\gitrepo{\juliaimages}{\imageaxes}}
\newcommand{\repoimagemetadata}{\gitrepo{\juliaimages}{\imagemetadata}}
\newcommand{\repoimagefiltering}{\gitrepo{\juliaimages}{\imagefiltering}}
\newcommand{\repoimagefeature}{\gitrepo{\juliaimages}{\imagefeature}}
\newcommand{\repoimagemorphology}{\gitrepo{\juliaimages}{\imagemorphology}}
\newcommand{\repoimagetracking}{\gitrepo{\juliaimages}{\imagetracking}}
\newcommand{\repodipcode}{\gitrepo{Johnnychen94}{Digital-Image-Processing-Gonzalez}}
\newcommand{\repodeeplearningtutorial}{\gitrepo{Johnnychen94}{DeepLearning\_Tutorial}}
\newcommand{\reposcikitimage}{\gitrepo{scikit-image}{scikit-image}}
\newcommand{\repojump}{\gitrepo{JuliaOpt}{\jump}}
\newcommand{\repogpuarrays}{\gitrepo{JuliaGPU}{\gpuarrays}}
\newcommand{\repoimages}{\gitrepo{\juliaimages}{\images}}
\newcommand{\repopycall}{\gitrepo{JuliaPy}{\pycall}}
\newcommand{\repoimagebinarization}{\gitrepo{zygmuntszpak}{\imagebinarization}}
\newcommand{\repohistogramthresholding}{\gitrepo{zygmuntszpak}{\histogramthresholding}}
\newcommand{\repoimagenoise}{\gitrepo{johnnychen94}{\imagenoise}}
\newcommand{\repoimagesgithubio}{\gitrepo{\juliaimages}{juliaimages.github.io}}
\newcommand{\repogsoctempdoc}{\gitrepo{johnnychen94}{GSoC2019\_Document}}
\newcommand{\repoimagedistance}{\gitrepo{\juliaimages}{\imagedistance}}
\newcommand{\repotestimages}{\gitrepo{\juliaimages}{\testimages}}
\newcommand{\repofixedpointnumbers}{\gitrepo{\juliamath}{\fixpointnumbers}}
\newcommand{\repoimageinpainting}{\gitrepo{\juliaimages}{\imageinpainting}}
\newcommand{\repoimageview}{\gitrepo{\juliaimages}{\imageview}}
\newcommand{\repoflux}{\gitrepo{\fluxml}{\flux}}


% other urls
\newcommand{\matlabimageprocessing}{\href{https://www.mathworks.com/products/image.html}{\sname{MATLAB Image Processing Toolbox}\xspace}}
\newcommand{\apicomparison}{\href{https://juliaimages.org/latest/api_comparison.html}{api comparison}\xspace}
\newcommand{\semanticversion}{\href{https://semver.org/}{Semantic Versioning}}

% ECNU
\newcommand{\ecnumath}{School of Mathematical Sciences, East China Normal University\xspace}

%% community people
\newcommand{\gitpeople}[2]{\href{https://github.com/#1}{\sname{#2}}\xspace}
\newcommand{\johnnychen}{\gitpeople{Johnnychen94}{Johnny Chen}}

\newcommand{\zygmunt}{\gitpeople{zygmuntszpak}{Zygmunt L. Szpak}}
\newcommand{\julio}{\gitpeople{juliohm}{J\'ulio Hoffimann}}
\newcommand{\evizero}{\gitpeople{Evizero}{Christof Stocker}}
\newcommand{\timholy}{\gitpeople{timholy}{Tim Holy}}
\newcommand{\mbauman}{\gitpeople{mbauman}{Matt Bauman}}
\newcommand{\mikeinnes}{\gitpeople{MikeInnes}{Mike J Innes}}


\renewcommand\Authfont{\fontsize{14.4}{14.4}\selectfont}
\renewcommand\Affilfont{\fontsize{12}{14.4}\selectfont}
\renewcommand\abstractname{Abstract}


\title{\huge Towards Better Images.jl Ecosystem}
\author{Jiuning Chen\mailto{johnnychen94@hotmail.com}}
\affil{Github: \href{https://github.com/johnnychen94}{Johnnychen94} \quad Slack: Johnny Chen\\
Mentors: \zygmunt, \julio, \timholy}
\date{\today}

\begin{document}
\maketitle

\begin{abstract}
    This project aims to achieve a better ecosystem for \href{https://juliaimages.org/latest/}{\images}, an image-processing toolbox in \href{https://julialang.org/}{\langjulia}. Main contributions consist of user-friendly documentation of \images{} ecosystem, developer manual, and more consistent, robust, and extensible APIs. Moreover, this project also serves as a subproject to \images{} \version{1.0} milestone.
\end{abstract}

\noindent The author, who began his contribution to \images{} since \date{Aug 2018}, is currently a third-year graduate student in the School of Mathematical Sciences, East China Normal University, Shanghai. His current research interests are image processing and computer vision, convex optimization, and machine learning. More information about him is listed in \cref{sec:about_author}.\par

\renewcommand\contentsname{Table of Contents}
\tableofcontents

\newpage

% !TEX root = ./main.tex

\section{Project Description}\label{sec:project}

\repoimages{} is a \langjulia image-processing toolbox, including several packages such as: \repoimagecore, \repoimagetransformations, \repoimageaxes. It provides a collection of out-of-box functions\footnote{An overview of currently implemented image-processing functionalities is shown at \apicomparison.} to do image processing tasks just like \reposcikitimage{} and \matlabimageprocessing{} do.

However, despite of the not yet benchmarked performance, this toolbox at present is still not friendly to both users and developers. Unlike other mature julia packages such as \repojump and \repogpuarrays, \images{} requires potential users and developers to understand the very details of its mechanism and architecture, and this becomes even harder for them without comprehensive documentation on it. Under this circumstance, many image-processing researchers are still using \langpython and \langmatlab for their daily work.

Some apparent causes for its poor usability are:
\begin{itemize}
    \item there're few demos or recipes in \images{} for new users to start with;
    \item the API varies greatly across Images.jl submodules, and can be unintuitive to the non-experts;
    \item there's no image-processing-specific style guide on naming and programming, except the \langjulia{} \href{https://docs.julialang.org/en/v1/manual/style-guide/}{style guide};
    \item there're too many temporary helper functions defined everywhere;
    \item \images{} is an ecosystem but it lacks of a comprehensive illustration of its packages;
    \item coverage of trait functions are not fully tested.
\end{itemize}
Fundamentally this is because that it is still in the progress of finding the most suitable programming style to process images using \langjulia.

Fortunately the problem is well-concerned in the community. Issues such as
\begin{itemize}
    \item \issue{\juliaimages}{\imagecore}{63} and \issue{\juliamath}{\fixedpointnumbers}{41} -- how to deal with overflow behavior of default \mintinline{julia}{N0f8} type?
    \item \issue{\juliaimages}{\images}{766} -- Use \mintinline{julia}{channelview} as possible as we can?
    \item \issue{\juliaimages}{\images}{767} -- Towards consistent style, part 1: a naming guide
    \item \issue{\juliaimages}{\images}{772} -- Revisiting the Images API
    \item \issue{zygmuntszpak}{\imagebinarization}{23} -- What's the appropriate argument order?
    \item \issue{zygmuntszpak}{\imagebinarization}{24} -- Export limited number of symbols?
\end{itemize}
discuss the coding styles and programming practice in the most generic way. Packages such as \repohistogramthresholding and \repoimagebinarization are examples that validate the effectiveness and usefulness of style consensus reached in those issues. For instance, in \imagebinarization, one could binarize an image using any implemented methods\footnote{At the time of writing, there're 12 methods implemented.} with one unified API:
\mint{julia}{binarize(::BinarizationAlgorithm, ::AbstractArray{T,2}) where {T}}

With these existing work, it's in the right time to revisit the whole \images{} ecosystem and head towards a more easy-to-use \images{} package. This project aims to solve this problem by:
\begin{enumerate}
    \item providing more comprehensive and integrated documentation on both style guide and ecosystem illustration,
    \item pruning codebase of the ecosystem according to the provided documentation
\end{enumerate}
Writing demos of \images{} is not included in this project since it belongs to a totally different project. Trait functions will be examined carefully to support high-level API design.

Basically, this is a project on documentation and code refactoring, and it is also a sub-project to \images{} \sname{v1.0} milestone. Potentially involved packages are:
\begin{itemize}
    \item \textbf{user entrance:} \repoimages
    \item \textbf{core packages:} \repoimagecore, \repoimageaxes and \repoimagemetadata
    \item \textbf{application packages:} \repoimagemorphology, \repoimagetransformations, \repoimagedistance and \repoimagefiltering
    \item \textbf{new packages\footnote{These packages are not yet imported by \images.}:} \repoimagebinarization, \repohistogramthresholding, \repoimageinpainting
\end{itemize}
High-level packages (e.g., \repoimagetracking) and ploting packages (e.g., \repoimageview) are not under consideration of this project.\par

This project will have many side effects:
\begin{itemize}
    \item partially rewritting of \href{https://juliaimages.org}{user documentation} in a more meaningful way;
    \item potential bug reports and patches to all related \langjulia repositories;
    \item introduction of a new image processing packages, \sname{ImageEdge.jl}, to place legacy methods in \images
    \item introduction of a new image denoising package, \repoimagenoise, as a concept-validation experimental field.\footnote{This is the author's research field.}
\end{itemize}


% !TEX root = ./main.tex

\section{Delivery Schedule}\label{sec:delivery}

This project is delivered in two stages: \textbf{documenting} and \textbf{pruning}. Documenting and recording in the first stage serves as the preparation for the second stage's pruning work of \images{} ecosystem.\par

Concerning the GSoC timeline, documenting stage begins from \date{April 22}\footnote{Although the coding officially begins from \date{May 27}, the author will start this project as soon as he's available.} to \date{June 24} (weeks 1-10), and the pruning stage starts from \date{July 1} to \date{August 26} (weeks 12-22). Week 11 serves as a buffer week. \textsf{Phase 1} evaluates the documentation work, and \textsf{Phase 2} and \textsf{Final} evaluates the pruning stage.

\subsection{Documenting}\label{subsec:documentation}

\subsubsection*{Stage Expectations}

The primary purpose of this stage is to provide trackable records for the next stage's pruning work. There'll be three types of records generated in this stage: \textbf{ecosystem documentation}, \textbf{developer manual}, and \textbf{RFCs} (Request For Comments). \par

Ecosystem documentation illustrates the scope of image ecosystem and relationships between different relevant packages; it helps users and developers to understand this ecosystem and its fundamental principles quickly. Developer manual consists of the style guide, best practice as well as other related community-operating rules; it gives a documented reference to developers to solve potential conflicts. RFCs with detailed lists of API changes and porting operations are proposed as trackable records for the pruning work in next stage. \par

\subsubsection*{Stage Workflow}

This stage is divided into two periods: \textbf{discussion period} and \textbf{RFC drafting period}. Ideally, this stage ends after the \textsf{Phase 1 Evaluation} with regard to GSoC timeline. However, since a lot of repositories are involved in this project, which makes the timeline hard to be sticked to, the following timeline serves in a flexible way.\par

The discussion period begins from \date{April 22} to \date{June 9} (weeks 1 to 7). In this period the community shares ideas and thoughts on the future of APIs and on best practices.\par

In the beginning of the descussion period, an ecosystem documentation be add to \repoimagesgithubio{} as soon as possible to reach a consensus on the future of \images{}, this consensus shall serve as the fundamental principle to all future discussion and development. Ideally, the current \images{} maintainer, i.e., \timholy, is supposed to participate in.\footnote{In case of maintainer being busy on other work, the author will draft a document based on his understanding and post it to the maintainer to get a feedback.}\par

The RFC drafting period begins from \date{May 27} to \date{June 23} (weeks 6 to 9). Based on previous discussions, one or more RFCs are drafted and discussed in this period. The last week of this stage is used for evaluation, merge, and announcement.\par

From weeks 1 to 7, many discussions happen simultaneously in the following way:
\begin{enumerate}
    \item \textbf{Code Review:} dig into source codes of repositories of images ecosystem to find anything that's likely in need of changing. Other mature \langjulia{} packages, and image-processing libraries in other languages such as \reposcikitimage{} and \matlabimageprocessing{} are references.
    \item \textbf{Issue Open:} open an issue for anything worth a discussion, e.g., legacy codes, misplaced codes, codes with bad practice, and undocumented practices and decisions.
    \item \textbf{Decision Making:} The conventional rules are taken: a decision is made when consensus is reached, otherwise the current maintainer of \images{} make the decision. If a decision can't be made before June 16 (Week 8), it'll be dropped as future work.
    \item \textbf{Record:} all approved, rejected, and future-work proposals are documented in a temporary repository - \repogsoctempdoc{}. Developer manual is drafted to \repoimagesgithubio{} when there're enough decisions made.
\end{enumerate}

From weeks 6 to 9, RFC drafting\footnote{A \href{https://github.com/tensorflow/community/blob/master/rfcs/yyyymmdd-rfc-template.md}{RFC Template} is available in the Tensorflow community, and \href{https://github.com/tensorflow/community/blob/master/rfcs/20180827-api-names.md}{20180827-api-names.md} is a good API-renaming RFC example.} happen simultaneously in the following way:
\begin{enumerate}
    \item \textbf{Code Review:} for each approved proposal, find all involved code pieces, and give a solution to it according to the developer manual. The principle of code review is to rigorously stick to decisions made in the discussion period -- either there's one principle or no principle.
    \item \textbf{RFC Post:} post the draft-version of RFC in \repogsoctempdoc{}.
    \item \textbf{RFC Review:} if there's an issue with any item in the proposed RFC, suspend the related items and go back to the discussion workflow until a decision is made.
    \item \textbf{RFC Approval and Announcement:} After approval of RFCs, they are merged to \repoimagesgithubio{} as records and announced to the community via slack and discourse. RFC merge and announcement only happen in the last two weeks in case there're more items to be added.
\end{enumerate}
RFC details on how the codebase is pruned is described in \cref{subsec:prune}.

\subsubsection*{Stage Evaluation}

Four items are evaluated at the end of this stage, i.e., \textsf{Phase 1 Evaluation}:
\begin{itemize}
    \item 2/10: activity on issues and discussions
    \item 2/10: ecosystem documentation
    \item 3/10: developer manual
    \item 3/10: RFCs
\end{itemize}
A score of 6/10 stands for \textsf{Evaluation Pass}.
% end of subsection

\subsection{Pruning Codebase}\label{subsec:prune}
After the RFCs being approved and announced to the community, the pruning stage begins. Ideally, this stage begins from \date{July 1} to \date{August 26} (weeks 12-22).

\subsubsection*{Stage Expectations}

The pruning stage is to clean the codebase according to the RFC operation guide. There'll be three types of pruning work:
\begin{itemize}
    \item symbol renaming, move, and removal -- backward incompatible
    \item API changes -- backward incompatible
    \item API enhancement -- backward compatible
\end{itemize}
For the ease of tracking pruning progress, a project/milestone is set in \repoimages{} to track the progress, and each pruning PR/issue is assigned a tag. \par

\subsubsection*{Stage Workflow}

Challenges during this stage are backward incompatibility and complex package dependencies. This section focus on strategies to address these. \par

One strategy of the pruning work is to start from packages with the least dependencies to that with the most dependencies. Using terms from \cref{sec:project}, we start from \sname{core packages} (e.g., \repoimagecore{}), to \sname{application packages} (e.g., \repoimagetransformations{}), and finally to the \sname{user-entrance package}, i.e., \repoimages{}. \sname{New packages} are easy to be handled since they're not officially included in \images{} ecosystem yet. \par

\newcommand{\packageA}{package \sname{A}\xspace}
\newcommand{\packageB}{package \sname{B}\xspace}
Another strategy is to do all the pruning work in separate branches to reduce the influence brought by its backward incompatibility. In other words, the workflow of pruning work is:
\begin{enumerate}
    \item create a separate branch \sname{api-prune} in each involved repositories, set up CI enviroment.
    \item port all methods and symbols in separate branches -- backward incompatible
    \item merge each branch into \sname{master}, and tag a minor version to each repository.
    \item freeze minor version for one or two months to let downstream packages upgrade their codebase. In the meantime, do backward compatible API enhancement
    \item remove deprecated symbols, methods and their tests, tag a minor version
\end{enumerate}
For time reason, steps 1-3 are counted as a part of GSoC project, and steps 4-5 belong to future work. Since lots of repositories are involved in this project, it's highly possible that this project ends in step 2.\par

Porting methods from \packageA to \packageB takes the following routine:
\begin{enumerate}
    \item implement new methods and unit tests in \packageB
    \item in \packageA, move methods to a separate \mintinline{julia}{deprecated.jl} file and deprecate them; these codes will be deleted after at least two minor releases.
\end{enumerate}

\subsubsection*{Stage Evaluation}

With regard to GSoC timeline, this stage includes the \sname{Phase 2} and \sname{Phase final} evaluations. Two attributes can be uses to evaluate the progress: milestone progress (percentage of merged PRs) and absolute number of opened PRs; the former helps evaluate the completness of this project and the latter helps evaluate the workload and difficulty of this project.\footnote{What we want is not to make the project pass but instead, to make \images{} a better ecosystem.}

The \sname{Phase 2} evaluation shall focus on checking if the pruning work begins as expected. The \sname{Phase final} evaluation shall focus on checking if the major part of pruning work is done.


% !TEX root = ./main.tex

\section{About the Author}\label{sec:about_author}

My name is \textsf{Jiuning Chen}, a third-year student in the School of Mathematical Sciences, East China Normal University, Shanghai. I'm currently doing research related to image processing, computer vision, convex optimization, and machine learning. It's worth noting that I got my bachelor's degree as a philosophy student, this experience makes me a good candidate to think about the future of \images{} ecosystem from a more generic and broader view -- I have good feelings about what needs improvement and what doesn't.\par

50+ hours per week can be guaranteed on this project since there's no other internship or vacation plan this summer. After GSoC, at least 20+ hours per week can be guaranteed to continue the long-term \images{} \version{1.0} project since I've passed the Ph.D. candidate examination in ECNU -- no compulsory work in the first academic year. \par

I learned \langjulia in Aug 2018 and become a contributor to \langjulia community since Oct 2018, and plan to make more contributions as a member of this community. Currently I'm one of the developers in \repoimagesgithubio{}.

\subsection*{Contribution to the community}
My main interest is to improve the \images, in the meantime, there are some small patches to \flux{} and \textsf{julia}. The major part of my contribution to \textsf{\juliaimages} is on issues: 25\% pull requests, 47\% issues, 19\% commits, 9\% code reviews. This is because I don't think \images{} is at its good status for new features until we pull this GSoC project off.

My previous PRs can help proving that I'm qualified to make progress on this project. Unless explicitly noted, all following PRs are non-trivial and merged.
\begin{itemize}
    \item Enhancement: \pullrequest{\julialang}{julia}{29626}\footnote{I only learned \langjulia{} for two months at that moment -- a proof of my good understanding on julia}\\
      {\small
      Support \mintinline{julia}{repeat} at any dimension as an enhancement\\
      reviewed by \mbauman
      }
    \item Feature: \pullrequest{\juliaimages}{\imagetransformations}{58}\\
      {\small
      Implement a robust \mintinline{julia}{imrotate} function\\
      reviewed by \evizero and \timholy
      }
    \item Enhancement: \pullrequest{\juliaimages}{\imagetransformations}{59}\\
      {\small
      Enhance \mintinline{julia}{imresize} with more delicate dispatch design\\
      Rewrite all the test cases of \mintinline{julia}{imresize}\\
      reviewed by \evizero and \timholy
      }
    \item Patch: \pullrequest{\juliaimages}{\imagedistance}{8}\\
      {\small
      fix a module design bug caused by misunderstanding of method overriding\\
      reviewed by \julio
      }
    \item Patch: \pullrequest{\juliaimages}{\imagesgithubio}{50} \\
      {\small
      fix the paralyzed autodeploy problem\\
      reviewed by \evizero
      }
    \item Documentation: \pullrequest{\juliaimages}{\testimages}{35}\\
      {\small
      initialize docstring of \mintinline{julia}{testimage}, which is the only exported function\\
      reviewed by \timholy
      }
    \item Documentation(RFC, pending): \pullrequest{zygmuntszpak}{\imagebinarization}{25}\\
      {\small
      redesign the docstring structure of all codes in \imagebinarization\\
      this PR serves as a first attempt to \issue{\juliaimages}{\images}{790} -- Towards consistent style, part 2: a documentation guide\\
      not yet reviewed because the maintainers are busy lately
      }
    \item Patch(pending): \pullrequest{\fluxml}{\flux}{710}\\
      {\small
      reimplement \mintinline{julia}{activations} to replace the current broken one\\
      reviewed by \mikeinnes
      }
    \item Patch: \pullrequest{\fluxml}{\flux}{372}\\
      {\small
      restrict argument type of \mintinline{julia}{Dense}, initialize test cases for \mintinline{julia}{layers/basic.jl}\\
      this PR is quite trivial but necessary\\
      reviewed by \mikeinnes
      }
    \item Patch: \pullrequest{\fluxml}{\flux}{371}\\
      {\small
      add support for \textsf{$\approx$} and fix dispatch bug on \textsf{$==$} \\
      reviewed by \mikeinnes
      }
\end{itemize}
I've also authored some related issues:
\begin{itemize}
  \item \issue{\juliaimages}{\images}{790} -- Towards consistent style, part 2: a documentation guide
  \item \issue{zygmuntszpak}{\imagebinarization}{24} -- Export limited number of symbols?
  \item \issue{\juliaimages}{\imagedistance}{7} -- Redesign the type hierarchy
  \item \issue{\juliaimages}{\images}{766} -- Use \mintinline{julia}{channelview} as possible as we can?
  \item \issue{\juliaimages}{\images}{765} -- Move \mintinline{julia}{difftype} to \imagecore
\end{itemize}

\subsection*{Other Experiences}
  \begin{itemize}
      \item Independently set up the whole self-hosted research platform from scratch for my supervisor's laboratory\\
        {\small
        5000 lines of markdown documentation is written by myself.
        }
      \item \textit{De facto} maintainer of the deep learning servers of the School of Mathematical Sciences, and that of a laboratory in Computer Sciences Department;
      \item Proudly created and maintained the \href{http://math.ecnu.edu.cn/~fli/}{homepage} of my supervisor.
      \item Head teaching assistant of courses: ``\href{http://math.ecnu.edu.cn/~fli/Teaching/DeepLearning/Fall2018/index.html}{Deep Learning and Action (Fall 2018)}''  and ``\href{http://math.ecnu.edu.cn/~fli/Teaching/DigitalImageProcessing/Spring2019/index.html}{Digital Image Processing (Spring 2019)}''.
  \end{itemize}


\subsection*{Self Evaluation}
\begin{itemize}
    \item \textbf{Mathematics \& Image Processing (8/10)}: my current research is on image denoising using variational model and deep learning;
    \item \textbf{Linux (7/10)}: heavy usage of docker, bash, git and vim in my daily work to maintain the servers;
    \item \textbf{Matlab (8/10)}: the only programming language used throughout my early-stage of research;
    \item \textbf{Julia (7/10)}: fully understand and stick to the philosophy of \langjulia, but lack of real and large project experience;
    \item \textbf{Related packages (6/10)}: familiar with other image-processing packages but haven't dig into the source code of them yet.
\end{itemize}

\subsection*{Education Background}

\begin{description}
    \item[2016-Present (Postgraduate)]Study on image processing and computer vision in School of Mathematical Sciences, East China Normal University, and supervised by Prof. \href{http://math.ecnu.edu.cn/~fli/}{\textsf{Fang Li}}.
    \item[2013-2016 (Undergraduate)] Bachelor of philosophy, Department of Philosophy, Shanghai University.
    \item[2011-2013 (Undergraduate)] Study on metal material in School of Material Sciences, Shanghai University.
\end{description}


\end{document}
