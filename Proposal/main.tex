% !TEX root = ./main.tex

\documentclass[12pt, a4paper]{article}
\usepackage[utf8]{inputenc}
\usepackage[colorlinks=true,urlcolor=blue,citecolor=SpringGreen4,bookmarks=true]{hyperref}
\usepackage{marvosym} % \Email
\usepackage{cleveref}
\usepackage{xspace}
\usepackage{minted}

% !TEX root = ./main.tex

\newcommand{\todo}[1]{\textcolor{red}{ TODO: (#1)}}
\newcommand{\ask}[1]{\textcolor{red}{ Question: #1}\par}

\newcommand{\mailto}[1]{\href{mailto:#1}{\textsuperscript{\Email}}}
\newcommand{\version}[1]{v$#1$\xspace}

\newcommand{\sname}[1]{\textsf{#1}} % special name

% programming
\newcommand{\langmatlab}{\sname{MATLAB}\xspace}
\newcommand{\langjulia}{\sname{Julia}\xspace}
\newcommand{\langpython}{\sname{Python}\xspace}

% github
\newcommand{\pullrequest}[3]{\href{https://github.com/#1/#2/pull/#3}{\sname{#1/#2\##3}}}
\newcommand{\issue}[3]{\href{https://github.com/#1/#2/pull/#3}{\sname{#1/#2\##3}}}

% organization names
\newcommand{\juliaimages}{JuliaImages}
\newcommand{\julialang}{JuliaLang}
\newcommand{\fluxml}{FluxML}
\newcommand{\juliamath}{JuliaMath}

% julia repository names
\newcommand{\reponame}[1]{#1}
\newcommand{\imagetransformations}{\reponame{ImageTransformations.jl}}
\newcommand{\imagecore}{\reponame{ImageCore.jl}}
\newcommand{\imageaxes}{\reponame{ImageAxes.jl}}
\newcommand{\imagemetadata}{\reponame{ImageMetadata.jl}}
\newcommand{\imagefiltering}{\reponame{ImageFiltering.jl}}
\newcommand{\imagemorphology}{\reponame{ImageMorphology.jl}}
\newcommand{\imagetracking}{\reponame{ImageTracking.jl}}
\newcommand{\images}{\reponame{Images.jl}}
\newcommand{\flux}{\reponame{Flux.jl}}
\newcommand{\jump}{\reponame{JuMP.jl}}
\newcommand{\gpuarrays}{\reponame{GPUArrays.jl}}
\newcommand{\pycall}{\reponame{PyCall.jl}}
\newcommand{\imagebinarization}{\reponame{ImageBinarization.jl}}
\newcommand{\histogramthresholding}{\reponame{HistogramThresholding.jl}}
\newcommand{\imagenoise}{\reponame{ImageNoise.jl}}
\newcommand{\imagedistance}{\reponame{ImageDistances.jl}}
\newcommand{\imagesgithubio}{\reponame{juliaimages.github.io}}
\newcommand{\testimages}{\reponame{TestImages.jl}}
\newcommand{\fixedpointnumbers}{\reponame{FixedPointNumbers.jl}}
\newcommand{\imageinpainting}{\reponame{ImageInpainting.jl}}
\newcommand{\imageview}{\reponame{ImageView.jl}}


% repository urls
\newcommand{\gitrepo}[2]{\href{https://github.com/#1/#2}{\sname{#2}}\xspace}
\newcommand{\repoimagetransformations}{\gitrepo{\juliaimages}{\imagetransformations}}
\newcommand{\repoimagecore}{\gitrepo{\juliaimages}{\imagecore}}
\newcommand{\repoimageaxes}{\gitrepo{\juliaimages}{\imageaxes}}
\newcommand{\repoimagemetadata}{\gitrepo{\juliaimages}{\imagemetadata}}
\newcommand{\repoimagefiltering}{\gitrepo{\juliaimages}{\imagefiltering}}
\newcommand{\repoimagemorphology}{\gitrepo{\juliaimages}{\imagemorphology}}
\newcommand{\repoimagetracking}{\gitrepo{\juliaimages}{\imagetracking}}
\newcommand{\repodipcode}{\gitrepo{Johnnychen94}{Digital-Image-Processing-Gonzalez}}
\newcommand{\repodeeplearningtutorial}{\gitrepo{Johnnychen94}{DeepLearning\_Tutorial}}
\newcommand{\reposcikitimage}{\gitrepo{scikit-image}{scikit-image}}
\newcommand{\repojump}{\gitrepo{JuliaOpt}{\jump}}
\newcommand{\repogpuarrays}{\gitrepo{JuliaGPU}{\gpuarrays}}
\newcommand{\repoimages}{\gitrepo{\juliaimages}{\images}}
\newcommand{\repopycall}{\gitrepo{JuliaPy}{\pycall}}
\newcommand{\repoimagebinarization}{\gitrepo{zygmuntszpak}{\imagebinarization}}
\newcommand{\repohistogramthresholding}{\gitrepo{zygmuntszpak}{\histogramthresholding}}
\newcommand{\repoimagenoise}{\gitrepo{johnnychen94}{\imagenoise}}
\newcommand{\repoimagesgithubio}{\gitrepo{\juliaimages}{juliaimages.github.io}}
\newcommand{\repogsoctempdoc}{\gitrepo{johnnychen94}{GSoC2019\_Document}}
\newcommand{\repoimagedistance}{\gitrepo{\juliaimages}{\imagedistance}}
\newcommand{\repotestimages}{\gitrepo{\juliaimages}{\testimages}}
\newcommand{\repofixedpointnumbers}{\gitrepo{\juliamath}{\fixpointnumbers}}
\newcommand{\repoimageinpainting}{\gitrepo{\juliaimages}{\imageinpainting}}
\newcommand{\repoimageview}{\gitrepo{\juliaimages}{\imageview}}


% other urls
\newcommand{\matlabimageprocessing}{\href{https://www.mathworks.com/products/image.html}{\sname{MATLAB Image Processing Toolbox}\xspace}}
\newcommand{\apicomparison}{\href{https://juliaimages.org/latest/api_comparison.html}{api comparison}\xspace}
\newcommand{\semanticversion}{\href{https://semver.org/}{Semantic Versioning}}

% ECNU
\newcommand{\ecnumath}{School of Mathematical Sciences, East China Normal University\xspace}

%% community people
\newcommand{\gitpeople}[2]{\href{https://github.com/#1}{\sname{#2}}\xspace}
\newcommand{\johnnychen}{\gitpeople{Johnnychen94}{Johnny Chen}}

\newcommand{\zygmunt}{\gitpeople{zygmuntszpak}{Zygmunt L. Szpak}}
\newcommand{\julio}{\gitpeople{juliohm}{J\'ulio Hoffimann}}
\newcommand{\evizero}{\gitpeople{Evizero}{Christof Stocker}}
\newcommand{\timholy}{\gitpeople{timholy}{Tim Holy}}
\newcommand{\mbauman}{\gitpeople{mbauman}{Matt Bauman}}
\newcommand{\mikeinnes}{\gitpeople{MikeInnes}{Mike J Innes}}


\title{GSoC 2019 Project\\Towards Better Images Ecosystem}
\author{Jiuning Chen\mailto{johnnychen94@hotmail.com}\\
Github: \href{https://github.com/johnnychen94}{Johnnychen94}}
\date{\today}

\begin{document}
\maketitle

\renewcommand\abstractname{Abstract}
\begin{abstract}
    This project aims to achieve a better ecosystem for \href{https://juliaimages.org/latest/}{\images}, an image-processing toolbox in \href{https://julialang.org/}{\langjulia}. Main contributions consist of definition of \images ecosystem and its scope, a manual for developers, and a more consistent API.
\end{abstract}

\noindent The author is currently a third-year graduate student and Ph.D candidate in School of Mathematical Sciences, East China Normal University, Shanghai, China. His current research interests are image processing and computer vision, convex optimization, and machine learning. More information about him is listed in \cref{sec:about_author}.\par

% !TEX root = ./main.tex

\section{Project Description}\label{sec:project}

\repoimages{} is a \langjulia image-processing toolbox that provides a collection of out-of-box functions\footnote{An overview of currently implemented image-processing functionalities is shown at \apicomparison.} to do image processing tasks just like \reposcikitimage{} and \matlabimageprocessing{} do.

\subsection*{Project Background}

However, despite the not yet benchmarked performance, this toolbox at present is still not friendly to both users and developers. Unlike other mature \langjulia{} packages such as \repojump and \repogpuarrays, \images{} requires potential users and developers to understand the very details of its mechanism and architecture, and this becomes even harder for them without comprehensive documentation on it. Under this circumstance, many image-processing researchers are still using \langpython and \langmatlab for their daily work.

Some apparent causes for its poor usability are:
\begin{itemize}
    \item there're few demos or recipes in \images{} for new users to start with;
    \item APIs vary greatly across Images.jl submodules, and are unintuitive to the non-experts;
    \item there's no image-processing-specific style guide on naming and programming, except the \langjulia{} \href{https://docs.julialang.org/en/v1/manual/style-guide/}{style guide};
    \item there're too many temporary helper functions defined everywhere;
    \item \images{} is an ecosystem but it lacks a comprehensive illustration of its packages;
    \item coverage of trait functions are not fully tested.
\end{itemize}
Fundamentally this is because the community is still in the progress of finding the most suitable programming style to process images using \langjulia.

Fortunately the problem is well-concerned in the community. Issues such as
\begin{itemize}
    \item \issue{\juliaimages}{\imagecore}{63} and \issue{\juliamath}{\fixedpointnumbers}{41} -- how to deal with overflow behavior of default \mintinline{julia}{N0f8} type?
    \item \issue{\juliaimages}{\images}{766} -- Use \mintinline{julia}{channelview} as possible as we can?
    \item \issue{\juliaimages}{\images}{767} -- Towards consistent style, part 1: a naming guide
    \item \issue{\juliaimages}{\images}{772} -- Revisiting the Images API
    \item \issue{\juliaimages}{\images}{790} -- Towards consistent style, part 1: a documentation guide
    \item \issue{zygmuntszpak}{\imagebinarization}{23} -- What's the appropriate argument order?
    \item \issue{zygmuntszpak}{\imagebinarization}{24} -- Export limited number of symbols?
\end{itemize}
discuss the coding styles and programming practice in the most generic way. Packages such as \repohistogramthresholding and \repoimagebinarization are examples that validate the effectiveness and usefulness of style consensus reached in those issues. For instance, in \imagebinarization, one could binarize an image using any implemented methods\footnote{At the time of writing, there're 12 methods implemented.} with one unified API:
\mint{julia}{binarize(::BinarizationAlgorithm, ::AbstractArray{T,2}) where {T}}

\subsection*{Project Expectation}
With these existing work, it's in the right time to revisit the whole \images{} ecosystem and head towards a more easy-to-use \images{} package, and more ambitiously, towards \images{} \version{1.0}. This project aims to solve this problem by:
\begin{enumerate}
    \item providing more comprehensive and integrated documentation on both style guide and ecosystem illustration, and drafting RFCs
    \item pruning codebase of the ecosystem according to the drafted RFCs
\end{enumerate}
Writing demos of \images{} is not included in this project since it belongs to a totally different project. Trait functions are examined carefully to support high-level API design.

This is a project on documentation and code refactoring to provide more consistent, robust and extensive APIs to both users and developers, and this is also a sub-project to \images{} \sname{v1.0} milestone. Potentially involved packages are:
\begin{itemize}
    \item \textbf{user entrance:} \repoimages
    \item \textbf{core packages:} \repoimagecore, \repoimageaxes and \repoimagemetadata
    \item \textbf{application packages:} \repoimagemorphology, \repoimagetransformations, \repoimagedistance and \repoimagefiltering
    \item \textbf{new packages\footnote{These packages are not yet imported by \images.}:} \repoimagebinarization, \repohistogramthresholding, \repoimageinpainting
\end{itemize}
Computer-vision packages (e.g., \repoimagetracking) and ploting packages (e.g., \repoimageview) are not under consideration of this project.\par

While doing the documenting and code refactoring work, this project has many side effects:
\begin{itemize}
    \item partially rewriting of \href{https://juliaimages.org}{user documentation} in a more meaningful way;
    \item potential bug reports and patches to all related \langjulia repositories;
    \item introduction of a new image processing package, \sname{ImageEdge.jl}, to place legacy methods in \images
    \item introduction of a new image denoising package, \repoimagenoise, as a concept-validation experimental field.\footnote{This is the author's research field.}
\end{itemize}
Note that these side effects are not counted as the purpose of this project.\par

Before introducing the details of delivery Schedule of this project, it's worth noting that workload and timeline of this project are hard to estimate due to two reasons:
\begin{itemize}
    \item this project can last for arbitrarily long time, and can contain an arbitrary number of issues and tasks; refactoring codebase can last forever.
    \item lots of repositories are get involved in this project; time delay in receiving inputs from others is significant.
\end{itemize}
Hence the purpose of this project isn't to make the ecosystem perfect, instead, it is to evolve an ecosystem that is significantly better in API and naming style, so that future development towards \images{} \version{1.0} is possible.


\section{Delivery and milestone}

% !TEX root = ./main.tex

\section{About the Author}\label{sec:about_author}

My name is \textsf{Jiuning Chen}, a third-year student in the School of Mathematical Sciences, East China Normal University, Shanghai. I'm currently doing research related to image processing, computer vision, convex optimization, and machine learning. Moreover, I'll be a Ph.D student in ECNU this fall. \par

40+ hours per week can be guaranteed on this project since there's no other internship or vacation plan this summer. After GSoC, at least 20+ hours per week can be guaranteed to continue the long-term \images{} \version{1.0} project -- no compulsory work in the first academic year as Ph.D student. \par

I learned \langjulia in Aug 2018 and become a contributor to \langjulia community since Oct 2018, and plan to make more contributions as a member of this community. To better pull this project off, I've been accepted as a member of \juliaimages.

\subsection*{Contribution to the community}
My main interest is to improve the \images, in the meantime, there are some small patches to \flux{} and \textsf{julia}. The major part of my contribution to \textsf{\juliaimages} is on issues: 26\% pull requests, 46\% issues, 18\% commits, 10\% code reviews. This is because I don't think \images{} is at its good status for new features until we achieve this GSoC project's expectations.

My previous PRs can help prove that I'm qualified to make progress on this project. Unless explicitly noted, all following PRs are non-trivial and merged.
\begin{itemize}
    \item Patch: \pullrequest{\julialang}{julia}{29626}\footnote{I only learned \langjulia{} for two months at that moment} \\
      {\small
      Support \mintinline{julia}{repeat} at any dimension
      }
    \item Feature: \pullrequest{\juliaimages}{\imagetransformations}{58} \\
      {\small
      Implement a robust \mintinline{julia}{imrotate} function
      }
    \item Enhancement: \pullrequest{\juliaimages}{\imagetransformations}{59} \\
      {\small
      Enhance \mintinline{julia}{imresize} with more delicate dispatch design\\
      Rewrite all legacy test cases of \mintinline{julia}{imresize}
      }
    \item Patch: \pullrequest{\juliaimages}{\imagedistance}{8} \\
      {\small
      fix a module design bug caused by misunderstanding of method overriding
      }
    \item Patch: \pullrequest{\juliaimages}{\imagesgithubio}{50} \\
      {\small
      fix the paralyzed autodeployment
      }
    \item Documentation: \pullrequest{\juliaimages}{\testimages}{35} \\
      {\small
      initialize docstring of \mintinline{julia}{testimage}, which is the only exported function
      }
    \item Documentation(RFC, pending): \pullrequest{zygmuntszpak}{\imagebinarization}{25}\\
      {\small
      redesign the docstring structure of all codes in \imagebinarization\\
      this PR serves as a first attempt to consistent documentation style
      }
    \item Patch: \pullrequest{\fluxml}{\flux}{710} \\
      {\small
      reimplement \mintinline{julia}{activations} to replace the current broken one
      }
    \item Patch: \pullrequest{\fluxml}{\flux}{372} \\
      {\small
      restrict argument type of \mintinline{julia}{Dense}, initialize test cases for \mintinline{julia}{layers/basic.jl}
      }
    \item Patch: \pullrequest{\fluxml}{\flux}{371} \\
      {\small
      add support for \textsf{$\approx$} and fix dispatch bug on \textsf{$==$}
      }
\end{itemize}
It's worth noting that I got my bachelor's degree as a philosophy student, this experience makes me a good candidate to think about the future of \images{} ecosystem from a more generic and broader view in a reflective and skeptical manner -- I have good feelings about what needs improvement and what doesn't. The issues I authored can help prove this:
\begin{itemize}
  \item \issue{\juliaimages}{\images}{792} -- reduce \images{} dependency complexity
  \item \issue{\juliaimages}{\images}{790} -- Towards consistent style, part 2: a documentation guide
  \item \issue{\juliaimages}{\images}{766} -- Use \mintinline{julia}{channelview} as possible as we can?
  \item \issue{\juliaimages}{\images}{765} -- Move \mintinline{julia}{difftype} to \imagecore
  \item \issue{zygmuntszpak}{\imagebinarization}{26} -- Filter specification: alternative to \mintinline{julia}{binarize}
  \item \issue{zygmuntszpak}{\imagebinarization}{24} -- Export limited number of symbols?
  \item \issue{\juliaimages}{\imagedistance}{7} -- Redesign the type hierarchy
\end{itemize}

\subsection*{Other Experiences}
  \begin{itemize}
      \item Independently set up the whole self-hosted research platform from scratch for my supervisor's laboratory\\
        {\small
        5000 lines of markdown user\&admin documentation is written by myself.
        }
      \item \textit{De facto} maintainer of deep learning servers of Math school, and that of a laboratory in CS department;
      \item Proudly created and maintained the \href{http://math.ecnu.edu.cn/~fli/}{homepage} of my supervisor.
      \item Head teaching assistant of courses: ``\href{http://math.ecnu.edu.cn/~fli/Teaching/DeepLearning/Fall2018/index.html}{Deep Learning and Action (Fall 2018)}''  and ``\href{http://math.ecnu.edu.cn/~fli/Teaching/DigitalImageProcessing/Spring2019/index.html}{Digital Image Processing (Spring 2019)}''.
  \end{itemize}


\subsection*{Self Evaluation}
\begin{itemize}
    \item \textbf{Mathematics \& Image Processing (8/10)}: my current research is on image denoising using variational model and deep learning;
    \item \textbf{Linux (7/10)}: heavy usage of docker, bash, git and vim in my daily work to maintain the servers;
    \item \textbf{Matlab (8/10)}: the only programming language used throughout my early-stage of research;
    \item \textbf{Julia (7/10)}: fully understand and stick to the philosophy of \langjulia, but lack of real and large project experience;
    \item \textbf{Related packages (6/10)}: familiar with other image-processing packages but haven't dig into the source code of them yet.
\end{itemize}

\subsection*{Education Background}

\begin{description}
    \item[2016-Present (Postgraduate)]Study on image processing and computer vision in School of Mathematical Sciences, East China Normal University, and supervised by Prof. \href{http://math.ecnu.edu.cn/~fli/}{\textsf{Fang Li}}.
    \item[2013-2016 (Undergraduate)] Bachelor of philosophy, Department of Philosophy, Shanghai University.
    \item[2011-2013 (Undergraduate)] Study on metal material in School of Material Sciences, Shanghai University.
\end{description}


\end{document}
