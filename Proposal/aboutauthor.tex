% !TEX root = ./main.tex

\section{About the Author}\label{sec:about_author}

My name is \textsf{Jiuning Chen}, a third-year student in the School of Mathematical Sciences, East China Normal University, Shanghai. I'm currently doing research related to image processing, computer vision, convex optimization, and machine learning. Moreover, I'll be a Ph.D student in ECNU this fall. \par

40+ hours per week can be guaranteed on this project since there's no other internship or vacation plan this summer. After GSoC, at least 20+ hours per week can be guaranteed to continue the long-term \images{} \version{1.0} project -- no compulsory work in the first academic year as Ph.D student. \par

I learned \langjulia in Aug 2018 and become a contributor to \langjulia community since Oct 2018, and plan to make more contributions as a member of this community. To better pull this project off, I've been accepted as a member of \juliaimages.

\subsection*{Contribution to the community}
My main interest is to improve the \images, in the meantime, there are some small patches to \flux{} and \textsf{julia}. The major part of my contribution to \textsf{\juliaimages} is on issues: 26\% pull requests, 46\% issues, 18\% commits, 10\% code reviews. This is because I don't think \images{} is at its good status for new features until we achieve this GSoC project's expectations.

My previous PRs can help prove that I'm qualified to make progress on this project. Unless explicitly noted, all following PRs are non-trivial and merged.
\begin{itemize}
    \item Patch: \pullrequest{\julialang}{julia}{29626}\footnote{I only learned \langjulia{} for two months at that moment} \\
      {\small
      Support \mintinline{julia}{repeat} at any dimension
      }
    \item Feature: \pullrequest{\juliaimages}{\imagetransformations}{58} \\
      {\small
      Implement a robust \mintinline{julia}{imrotate} function
      }
    \item Enhancement: \pullrequest{\juliaimages}{\imagetransformations}{59} \\
      {\small
      Enhance \mintinline{julia}{imresize} with more delicate dispatch design\\
      Rewrite all legacy test cases of \mintinline{julia}{imresize}
      }
    \item Patch: \pullrequest{\juliaimages}{\imagedistance}{8} \\
      {\small
      fix a module design bug caused by misunderstanding of method overriding
      }
    \item Patch: \pullrequest{\juliaimages}{\imagesgithubio}{50} \\
      {\small
      fix the paralyzed autodeployment
      }
    \item Documentation: \pullrequest{\juliaimages}{\testimages}{35} \\
      {\small
      initialize docstring of \mintinline{julia}{testimage}, which is the only exported function
      }
    \item Documentation(RFC, pending): \pullrequest{zygmuntszpak}{\imagebinarization}{25}\\
      {\small
      redesign the docstring structure of all codes in \imagebinarization\\
      this PR serves as a first attempt to consistent documentation style
      }
    \item Patch: \pullrequest{\fluxml}{\flux}{710} \\
      {\small
      reimplement \mintinline{julia}{activations} to replace the current broken one
      }
    \item Patch: \pullrequest{\fluxml}{\flux}{372} \\
      {\small
      restrict argument type of \mintinline{julia}{Dense}, initialize test cases for \mintinline{julia}{layers/basic.jl}
      }
    \item Patch: \pullrequest{\fluxml}{\flux}{371} \\
      {\small
      add support for \textsf{$\approx$} and fix dispatch bug on \textsf{$==$}
      }
\end{itemize}
It's worth noting that I got my bachelor's degree as a philosophy student, this experience makes me a good candidate to think about the future of \images{} ecosystem from a more generic and broader view in a reflective and skeptical manner -- I have good feelings about what needs improvement and what doesn't. The issues I authored can help prove this:
\begin{itemize}
  \item \issue{\juliaimages}{\images}{792} -- reduce \images{} dependency complexity
  \item \issue{\juliaimages}{\images}{790} -- Towards consistent style, part 2: a documentation guide
  \item \issue{\juliaimages}{\images}{766} -- Use \mintinline{julia}{channelview} as possible as we can?
  \item \issue{\juliaimages}{\images}{765} -- Move \mintinline{julia}{difftype} to \imagecore
  \item \issue{zygmuntszpak}{\imagebinarization}{26} -- Filter specification: alternative to \mintinline{julia}{binarize}
  \item \issue{zygmuntszpak}{\imagebinarization}{24} -- Export limited number of symbols?
  \item \issue{\juliaimages}{\imagedistance}{7} -- Redesign the type hierarchy
\end{itemize}

\subsection*{Other Experiences}
  \begin{itemize}
      \item Independently set up the whole self-hosted research platform from scratch for my supervisor's laboratory\\
        {\small
        5000 lines of markdown user\&admin documentation is written by myself.
        }
      \item \textit{De facto} maintainer of deep learning servers of Math school, and that of a laboratory in CS department;
      \item Proudly created and maintained the \href{http://math.ecnu.edu.cn/~fli/}{homepage} of my supervisor.
      \item Head teaching assistant of courses: ``\href{http://math.ecnu.edu.cn/~fli/Teaching/DeepLearning/Fall2018/index.html}{Deep Learning and Action (Fall 2018)}''  and ``\href{http://math.ecnu.edu.cn/~fli/Teaching/DigitalImageProcessing/Spring2019/index.html}{Digital Image Processing (Spring 2019)}''.
  \end{itemize}


\subsection*{Self Evaluation}
\begin{itemize}
    \item \textbf{Mathematics \& Image Processing (8/10)}: my current research is on image denoising using variational model and deep learning;
    \item \textbf{Linux (7/10)}: heavy usage of docker, bash, git and vim in my daily work to maintain the servers;
    \item \textbf{Matlab (8/10)}: the only programming language used throughout my early-stage of research;
    \item \textbf{Julia (7/10)}: fully understand and stick to the philosophy of \langjulia, but lack of real and large project experience;
    \item \textbf{Related packages (6/10)}: familiar with other image-processing packages but haven't dig into the source code of them yet.
\end{itemize}

\subsection*{Education Background}

\begin{description}
    \item[2016-Present (Postgraduate)]Study on image processing and computer vision in School of Mathematical Sciences, East China Normal University, and supervised by Prof. \href{http://math.ecnu.edu.cn/~fli/}{\textsf{Fang Li}}.
    \item[2013-2016 (Undergraduate)] Bachelor of philosophy, Department of Philosophy, Shanghai University.
    \item[2011-2013 (Undergraduate)] Study on metal material in School of Material Sciences, Shanghai University.
\end{description}
