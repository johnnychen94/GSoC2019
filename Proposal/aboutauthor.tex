% !TEX root = ./main.tex

\section{About the Author}\label{sec:about_author}

My name is \textsf{Jiuning Chen}. I'm currently doing research related to image processing, computer vision, convex optimization and machine learning. 40+ hours per week can be guaranteed on this project.

\subsection*{Programming Background}

I started to use \langmatlab to do research on image processing in the end of 2016, met  \langjulia after its \version{1.0} announcement, and learned \langpython during the Spring Festival of 2019. \par

Although my programming career is less than three years, however, I think I'm qualified to achieve the project expectations for the non-trivial contributions I've done to the \langjulia{} community:

\begin{itemize}
    \item PR(merged): \pullrequest{\juliaimages}{\imagetransformations}{58} reviewed by \evizero and \timholy;
    \item PR(merged): \pullrequest{\juliaimages}{\imagetransformations}{59} reviewed by \evizero and \timholy;
    \item PR(merged): \pullrequest{\juliaimages}{\imagedistance}{8} reviewed by \julio;
    \item PR(merged): \pullrequest{\juliaimages}{\testimages}{35} reviewed by \timholy;
    \item PR(merged): \pullrequest{\julialang}{julia}{29626} reviewed by \mbauman;
    \item PR(merged): \pullrequest{\fluxml}{\flux}{372} reviewed by \mikeinnes;
    \item PR(merged): \pullrequest{\fluxml}{\flux}{371} reviewed by \mikeinnes;
\end{itemize}
to members in the lab of my supervisor:
\begin{itemize}
    \item Independently set up the whole self-hosted research platform from scratch for my supervisor's laboratory\footnote{The platform includes but not limited to homepage, documents for users and administrators, server monitor, gitlab, jupyterhub, sharelatex, DNS servers, and VPN servers.};
    \item \textit{De facto} maintainer of the deep learning servers of the School of Mathematical Sciences, and that of a laboratory in Computer Sciences Department;
    \item Proudly create and maintain the homepage of my supervisor, prof. \href{http://math.ecnu.edu.cn/~fli/}{\textsf{Fang Li}};
\end{itemize}
and to undergraduate students in the university:
\begin{itemize}
    \item Head teaching assistant of courses of ``\href{http://math.ecnu.edu.cn/~fli/Teaching/DeepLearning/Fall2018/index.html}{Deep Learning and Action (Fall 2018)}''  and ``\href{http://math.ecnu.edu.cn/~fli/Teaching/DigitalImageProcessing/Spring2019/index.html}{Digital Image Processing (Spring 2019)}''.
    \item Unofficially mentor talented students with all the best programming practices I learned from the open-source community and from the English world\footnote{Most Chinese students are afraid of reading English since it's not their native language, however, almost all best materials are in forms of it. My role here is to learn and to preach.}.
\end{itemize}

The following is an informal self-evaluation to let you have a more structural overview of my skill:

\begin{itemize}
    \item \textbf{Mathematics \& Image Processing (8/10)}: my current research is on image denoising based on hybrid method of variational model and deep learning;
    \item \textbf{Linux (7/10)}: heavy usage of docker, bash, git and vim in my daily work to maintain the servers;
    \item \textbf{Matlab (8/10)}: the only programming language used throughout my early-stage of research;
    \item \textbf{Julia (6/10)}: fully understand and stick to the philosophy of \langjulia, but lack of real project experience;
    \item \textbf{Related packages (6/10)}: familiar with other image-processing packages but haven't dig into the source code of them yet;
\end{itemize}

\subsection*{Education Background}

\begin{description}
    \item[2016-Present (Postgraduate)]Study on image processing and computer vision in School of Mathematical Sciences, East China Normal University, and supervised by Prof. \href{http://math.ecnu.edu.cn/~fli/}{\textsf{Fang Li}}.
    \item[2013-2016 (Undergraduate)] Bachelor of philosophy, Department of Philosophy, Shanghai University.
    \item[2011-2013 (Undergraduate)] Study on metal material in School of Material Sciences, Shanghai University.
\end{description}
